\documentclass[a4paper,10pt,twocolumn,oneside]{article}
\setlength{\columnsep}{10pt}                                                                    %兩欄模式的間距
\setlength{\columnseprule}{0pt}                                                                %兩欄模式間格線粗細

\usepackage{amsthm}								%定義,例題
\usepackage{amssymb}
%\usepackage[margin=2cm]{5_Geometry}
\usepackage{fontspec}								%設定字體
\usepackage{color}
\usepackage[x11names]{xcolor}
\usepackage{listings}								%顯示code用的
%\usepackage[Glenn]{fncychap}						%排版,頁面模板
\usepackage{fancyhdr}								%設定頁首頁尾
\usepackage{graphicx}								%Graphic
\usepackage{enumerate}
\usepackage{enumitem}
\usepackage{titlesec}
\usepackage{amsmath}
\usepackage[CheckSingle, CJKmath]{xeCJK}
% \usepackage{CJKulem}

%\usepackage[T1]{fontenc}
\titlespacing{\section}{0cm}{0cm}{0cm}
\titlespacing{\subsection}{0cm}{0cm}{0cm}
\usepackage{amsmath, courier, listings, fancyhdr, graphicx}
\topmargin=0pt
\headsep=5pt
\textheight=780pt
\footskip=0pt
\voffset=-40pt
\textwidth=550pt
\marginparsep=0pt
\marginparwidth=0pt
\marginparpush=0pt
\oddsidemargin=0pt
\evensidemargin=0pt
\hoffset=-50pt

%\renewcommand\listfigurename{圖目錄}
%\renewcommand\listtablename{表目錄} 

%%%%%%%%%%%%%%%%%%%%%%%%%%%%%

\setmainfont [				%主要字型
    Path = .fonts/ttf/,
    UprightFont = *-Regular,
    BoldFont = *-Bold,
    ItalicFont = *-Italic
  ] {Consolas}

\setmonofont [        
    Path = .fonts/ttf/,
    UprightFont = *-Regular
  ] {Monaco}
%\setCJKmainfont{BabelStone Han}
%\setCJKmainfont{Source Han Sans}
%\setCJKmainfont{Consolas}			%中文字型
%\setmainfont{sourcecodepro}
\XeTeXlinebreaklocale "zh"						%中文自動換行
\XeTeXlinebreakskip = 0pt plus 1pt				%設定段落之間的距離
\setcounter{secnumdepth}{3}						%目錄顯示第三層

%%%%%%%%%%%%%%%%%%%%%%%%%%%%%
\makeatletter
\lst@CCPutMacro\lst@ProcessOther {"2D}{\lst@ttfamily{-{}}{-{}}}
\@empty\z@\@empty
\makeatother
\lstset{											% Code顯示
language=C++,										% the language of the code
basicstyle=\footnotesize\ttfamily, 						% the size of the fonts that are used for the code
%numbers=left,										% where to put the line-numbers
numberstyle=\footnotesize,						% the size of the fonts that are used for the line-numbers
stepnumber=1,										% the step between two line-numbers. If it's 1, each line  will be numbered
numbersep=5pt,										% how far the line-numbers are from the code
backgroundcolor=\color{white},					% choose the background color. You must add \usepackage{color}
showspaces=false,									% show spaces adding particular underscores
showstringspaces=false,							% underline spaces within strings
showtabs=false,									% show tabs within strings adding particular underscores
frame=false,											% adds a frame around the code
tabsize=2,											% sets default tabsize to 2 spaces
captionpos=b,										% sets the caption-position to bottom
breaklines=true,									% sets automatic line breaking
breakatwhitespace=false,							% sets if automatic breaks should only happen at whitespace
escapeinside={\%*}{*)},							% if you want to add a comment within your code
morekeywords={*},									% if you want to add more keywords to the set
keywordstyle=\bfseries\color{Blue1},
commentstyle=\itshape\color{Red4},
stringstyle=\itshape\color{Green4},
}

%%%%%%%%%%%%%%%%%%%%%%%%%%%%%

\begin{document}
\pagestyle{fancy}
\fancyfoot{}
%\fancyfoot[R]{\includegraphics[width=20pt]{ironwood.jpg}}
\fancyhead[L]{National Tsing Hua University GeomeTrick}
\fancyhead[R]{\thepage}
\renewcommand{\headrulewidth}{0.4pt}
\renewcommand{\contentsname}{Contents} 

\scriptsize
\tableofcontents
%%%%%%%%%%%%%%%%%%%%%%%%%%%%%

% \newpage

\section{Basic}
\subsection{.vimrc}
\lstinputlisting{1_Basic/.vimrc}

\subsection{Default}
\lstinputlisting{1_Basic/Default.cpp}

\subsection{IO Optimize}
\lstinputlisting{1_Basic/IO_Optimize.cpp}

\subsection{PBDS}
\lstinputlisting{1_Basic/PBDS.cpp}

\subsection{Random}
\lstinputlisting{1_Basic/Random.cpp}

\subsection{run.sh}
\lstinputlisting{1_Basic/run.sh}

\section{Graph}
\subsection{2 SAT}
\lstinputlisting{2_Graph/2_SAT.cpp}

\subsection{Biconnected Component}
\lstinputlisting{2_Graph/Biconnected_Component.cpp}

\subsection{Bridge connected Component}
\lstinputlisting{2_Graph/Bridge_Connected_Component.cpp}

\subsection{Bridge}
\lstinputlisting{2_Graph/Bridge.cpp}

\subsection{Close Vertices}
\lstinputlisting{2_Graph/Close_Vertices.cpp}

\subsection{Disjoint Set}
\lstinputlisting{2_Graph/Disjoint_Set.cpp}

\subsection{Heavy Light Decomposition}
\lstinputlisting{2_Graph/Heavy_Light_Decomposition.cpp}

\subsection{KSP}
\lstinputlisting{2_Graph/KSP.cpp}

\subsection{LCA}
\lstinputlisting{2_Graph/LCA.cpp}

\subsection{SCC Kosaraju}
\lstinputlisting{2_Graph/SCC_Kosaraju.cpp}

\subsection{SCC Tarjan}
\lstinputlisting{2_Graph/SCC_Tarjan.cpp}

\subsection{Tree Centroid}
\lstinputlisting{2_Graph/Tree_Centroid.cpp}

\section{Data Structure}
\subsection{2D BIT}
\lstinputlisting{3_Data_Structure/2D_BIT.cpp}

\subsection{2D Segment Tree}
\lstinputlisting{3_Data_Structure/2D_Segment_Tree.cpp}

\subsection{BIT}
\lstinputlisting{3_Data_Structure/BIT.cpp}

\subsection{LiChaoST.cpp}
\lstinputlisting{3_Data_Structure/LiChaoST.cpp}

\subsection{Segment Tree}
\lstinputlisting{3_Data_Structure/Segment_Tree.cpp}

\subsection{Sparse Table}
\lstinputlisting{3_Data_Structure/Sparse_Table.cpp}

\subsection{Treap}
\lstinputlisting{3_Data_Structure/Treap.cpp}

\subsection{ZKW Segment Tree}
\lstinputlisting{3_Data_Structure/ZKW_Segment_Tree.cpp}

\section{Flow}
\subsection{Bipartite Matching}
\lstinputlisting{4_Flow/Bipartite_Matching.cpp}

\subsection{Dinic}
\lstinputlisting{4_Flow/Dinic.cpp}

\subsection{KM}
\lstinputlisting{4_Flow/KM.cpp}

\subsection{MCMF}
\lstinputlisting{4_Flow/MCMF.cpp}

\section{Geometry}
\subsection{Basic 2D}
\lstinputlisting{5_Geometry/Basic_2D.cpp}

\subsection{BronKerbosch}
\lstinputlisting{5_Geometry/BronKerbosch.cpp}

\subsection{Convex Hull}
\lstinputlisting{5_Geometry/Convex_Hull.cpp}

\subsection{Dynamic Convex Hull}
\lstinputlisting{5_Geometry/Dynamic_Convex_Hull.cpp}

\subsection{Segmentation Intersection}
\lstinputlisting{5_Geometry/Segmentation_Intersection.cpp}

\section{Math}
\subsection{Big Int}
\lstinputlisting{6_Math/Big_Int.cpp}

\subsection{Extgcd}
\lstinputlisting{6_Math/Extgcd.cpp}

\subsection{Karatsuba}
\lstinputlisting{6_Math/Karatsuba.cpp}

\subsection{Linear Sieve}
\lstinputlisting{6_Math/Linear_Sieve.cpp}

\subsection{Matrix}
\lstinputlisting{6_Math/Matrix.cpp}

\subsection{Miller Rabin}
\lstinputlisting{6_Math/Miller_Rabin.cpp}

\subsection{NTT}
\lstinputlisting{6_Math/NTT.cpp}

\subsection{Pollard Rho}
\lstinputlisting{6_Math/Pollard_Rho.cpp}

\subsection{Theorem}
\begin{itemize}[leftmargin=*]
\setlength\itemsep{0.2em}
\item Pick's Theorem:\\
if a polygon has vertices with integer coordinates (lattice points), then the area is $i+\frac{1}{2}p−1$ where $i$ is the number of lattice points inside the polygon and $p$ is the number of lattice points on the perimeter of the polygon. i.e.
$$area(P)=i+\frac{1}{2}p-1$$

\end{itemize}

\section{String}
\subsection{AC}
\lstinputlisting{7_String/AC.cpp}
\subsection{Hash}
\lstinputlisting{7_String/Hash.cpp}

\subsection{KMP}
\lstinputlisting{7_String/KMP.cpp}

\subsection{Manacher}
\lstinputlisting{7_String/Manacher.cpp}

\subsection{SA}
\lstinputlisting{7_String/SA.cpp}

%\subsection{SA2}
%\lstinputlisting{7_String/SA2.cpp}

\subsection{SAIS}
\lstinputlisting{7_String/SAIS.cpp}

\subsection{Z}
\lstinputlisting{7_String/Z.cpp}

\section{Others}
\subsection{Mo}
\lstinputlisting{8_Others/Mo.cpp}

\subsection{Partial Ordering}
\lstinputlisting{8_Others/Partial_Ordering.cpp}


\end{document}